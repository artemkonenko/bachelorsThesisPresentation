%%%%%%%%%%%%%%%%%%%%%%%%%%%%%%%%%%%%%%%%%
% Beamer Presentation
% LaTeX Template
% Version 1.0 (10/11/12)
%
% This template has been downloaded from:
% http://www.LaTeXTemplates.com
%
% License:
% CC BY-NC-SA 3.0 (http://creativecommons.org/licenses/by-nc-sa/3.0/)
%
%%%%%%%%%%%%%%%%%%%%%%%%%%%%%%%%%%%%%%%%%

%----------------------------------------------------------------------------------------
%	PACKAGES AND THEMES
%----------------------------------------------------------------------------------------

\documentclass[utf8,russian]{beamer}

\mode<presentation> {

% The Beamer class comes with a number of default slide themes
% which change the colors and layouts of slides. Below this is a list
% of all the themes, uncomment each in turn to see what they look like.

%\usetheme{default}
%\usetheme{AnnArbor}
%\usetheme{Antibes}
%\usetheme{Bergen}
%\usetheme{Berkeley}
%\usetheme{Berlin}
%\usetheme{Boadilla}
%  \usetheme{CambridgeUS}
%\usetheme{Copenhagen}
%\usetheme{Darmstadt}
%\usetheme{Dresden}
%\usetheme{Frankfurt}
%\usetheme{Goettingen}
%\usetheme{Hannover}
%\usetheme{Ilmenau}
%\usetheme{JuanLesPins}
%\usetheme{Luebeck}
\usetheme{Madrid}
%\usetheme{Malmoe}
%\usetheme{Marburg}
%\usetheme{Montpellier}
%  \usetheme{PaloAlto}
%\usetheme{Pittsburgh}
%\usetheme{Rochester}
%\usetheme{Singapore}
%\usetheme{Szeged}
%\usetheme{Warsaw}

% As well as themes, the Beamer class has a number of color themes
% for any slide theme. Uncomment each of these in turn to see how it
% changes the colors of your current slide theme.

%\usecolortheme{albatross}
%\usecolortheme{beaver}
%\usecolortheme{beetle}
%\usecolortheme{crane}
%\usecolortheme{dolphin}
%\usecolortheme{dove}
%\usecolortheme{fly}
%\usecolortheme{lily}
%\usecolortheme{orchid}
%\usecolortheme{rose}
%\usecolortheme{seagull}
%\usecolortheme{seahorse}
%\usecolortheme{whale}
%\usecolortheme{wolverine}

%\setbeamertemplate{footline} % To remove the footer line in all slides uncomment this line
%\setbeamertemplate{footline}[page number] % To replace the footer line in all slides with a simple slide count uncomment this line

%\setbeamertemplate{navigation symbols}{} % To remove the navigation symbols from the bottom of all slides uncomment this line
}

\usepackage{graphicx} % Allows including images
\usepackage{booktabs} % Allows the use of \toprule, \midrule and \bottomrule in tables

\usepackage[utf8]{inputenc}
\usepackage[T2A]{fontenc}
\usepackage[russian,english]{babel}

\setbeamertemplate{navigation symbols}{} % To remove the navigation symbols from the bottom of all slides uncomment this line


\setbeamercolor{footline}{}
\setbeamertemplate{footline}{
  \leavevmode%
  \hbox{%
  \begin{beamercolorbox}[wd=.333333\paperwidth,ht=2.25ex,dp=1ex,center]{}%
    А. С. Коненко (Мехмат ЮФУ)
  \end{beamercolorbox}%
  \begin{beamercolorbox}[wd=.333333\paperwidth,ht=2.25ex,dp=1ex,center]{}%
    Ростов-на-Дону, 2015
  \end{beamercolorbox}%
  \begin{beamercolorbox}[wd=.333333\paperwidth,ht=2.25ex,dp=1ex,right]{}%
  Стр. \insertframenumber{} из \inserttotalframenumber \hspace*{2ex}
  \end{beamercolorbox}}%
  \vskip0pt%
}

%----------------------------------------------------------------------------------------
%	TITLE PAGE
%----------------------------------------------------------------------------------------

%\title[Анализ псевдонимов]{Оптимизация работы анализа псевдонимов в ОРС} % The short title appears at the bottom of every slide, the full title is only on the title page

%\author{Артём Коненко} % Your name
%\institute[МехМат] % Your institution as it will appear on the bottom of every slide, may be shorthand to save space
%{
%МехМат \\ % Your institution for the title page
%\medskip
%\textit{yadummer@gmail.com} % Your email address
%}
%\date{21.06.15} % \today % Date, can be changed to a custom date

\title{\small{Оптимизация работы анализа псевдонимов в ОРС}}
\vspace{15pt}%
\author{\small{%
А. С. Коненко\\%
\emph{Направление подготовки:}~Фундаментальная информатика и \\информационные технологии\\%
\emph{Научный руководитель:}~к.ф.-м.н., старший преподаватель Р.~Б.~Штейнберг}\\%
\vspace{15pt}%
    Южный федеральный университет\\
    Институт математики, механики и компьютерных наук
    имени~И.\,И.\,Воровича%
}
\date{\small{Ростов-на-Дону, 2015}}

\begin{document}

\begin{frame}
\titlepage % Print the title page as the first slide
\end{frame}

\begin{frame}
\frametitle{Содержание} % Table of contents slide, comment this block out to remove it
\tableofcontents % Throughout your presentation, if you choose to use \section{} and \subsection{} commands, these will automatically be printed on this slide as an overview of your presentation
\end{frame}

%----------------------------------------------------------------------------------------
%	PRESENTATION SLIDES
%----------------------------------------------------------------------------------------

%-----------------------------------------------------------------------------------------
%-----------------------------------------------------------------------------------------
\section{Постановка задачи}
%-----------------------------------------------------------------------------------------

\begin{frame}
\frametitle{Постановка задачи}
\begin{itemize}
  \item \textbf{Изучить существующие подходы к анализу псевдонимов.}
  \item \textbf{Исследовать возможности ускорения существующей реализации анализа псевдонимов.}  % Обнаружить узкие места в существующей реализации анализа псевдонимов.
  \item \textbf{Модифицировать существующие алгоритмы для ускорения анализа.} 
  \item \textbf{Протестировать работу новой реализации анализа и сравнить производительность.} 
\end{itemize}
\end{frame}

%-----------------------------------------------------------------------------------------
%-----------------------------------------------------------------------------------------
\section{Подходы к анализу псевдонимов}
%-----------------------------------------------------------------------------------------

\begin{frame}
\frametitle{Определения}

\begin{block}{Псевдонимы / Алиасы}
Два указателя a и b будем называть псевдонимами, если они указывают на одну и ту же область памяти.
\end{block}

\begin{block}{Контекст}
Контекстом вызова процедуры -- информация о взаимоотношение между всеми всеми указателями, которые находятся в области видимости данной процедуры при вызове её из данного места.
\end{block}

\end{frame}

%-----------------------------------------------------------------------------------------

\begin{frame}
\frametitle{Использование}

\begin{block}{Цель анализа}
Cтатически (на этапе компиляции) определить для каждой пары указателей и, возможно, контекста, что
\begin{itemize}
\item они всегда указывают в одну и ту же память -- $MustAlias$.
\item объекты, на которые они указывают пересекаются в памяти, но начинаются с различных адресов -- $PartialAlias$.
\item они могут указывать в одну область памяти -- $MayAlias$.
\item они всегда указывают в разные области памяти -- $NoAlias$.
\end{itemize}
\end{block}

\end{frame}

%-----------------------------------------------------------------------------------------

\begin{frame}
\frametitle{Применение}

\begin{block}{Для уточнения других анализов}
Результаты некоторых анализов можно уточнить, если знать взаимоотношения между данными, которые участвуют в анализе.
% example: анализ FrameDataSpecification - определение объема требуемой памяти. Если какой-то указатель всегда псевдоним, то и анализировать его не нужно.
\end{block}

\begin{block}{Для проверки выполнения предусловий применения преобразований}
У некоторых преобразований есть предусловия связанные с псевдонимами, которые разрешают применять их только если какие-то указатели точно псевдонимы или точно не псевдонимы.
% example: разрезание цикла или развертка и/или если два указателя всегда точно псевдонимы, то можно один заменить на второй и первый выкинуть.
\end{block}

\end{frame}

%-----------------------------------------------------------------------------------------

\begin{frame}
\frametitle{Классификация анализов псевдонимов}

\begin{block}{context-sensitive / context-insensitive}
Чувствительность к параметрам вызова анализируемой функции/процедуры.
\end{block}

\begin{block}{flow-sensitive / flow-insensitive}
Чувствительность к местам вхождений переданных указателей.
\end{block}

\begin{block}{field-sensitive / field-insensitive}
Способность различать разные поля одной структуры.
\end{block}

\begin{block}{type-based / flow-based}
Основанный на анализе типов или на анализе потока управления программы.
\end{block}

\end{frame}

%-----------------------------------------------------------------------------------------

\begin{frame}
\frametitle{Рассмотренные подходы к анализу псевдонимов}

\begin{block}{}
\begin{enumerate}
\item BasicAA -- Основной анализ использующийся в LLVM
\item Steensgaard’s algorithm -- Алгоритм межпроцедурного анализа с хорошей скоростью работы, но с большим числом упрощений.
\item Partial Transfer Functions -- Подход к анализу, который строит отображения из контекста вызова в контекст после вызова функции.
\end{enumerate}
\end{block}

\begin{block}{}
При наличии в компиляторе нескольких реализаций анализа, они могут использоватся по цепочке для построения общего неточного результата и уточнения критичных частей.
\end{block}
\end{frame}

%-----------------------------------------------------------------------------------------
%-----------------------------------------------------------------------------------------
\section{Существующий анализ псевдонимов в ОРС}
%-----------------------------------------------------------------------------------------

\begin{frame}
\frametitle{Существующий анализ псевдонимов в ОРС}

\begin{block}{}
Существующий в ОРС анализ псевдонимов осуществляет для каждой функции обход факторизованного графа потока управления с накоплением результатов анализа и анализирует вызовы функций для каждого контекста отдельно.
\end{block}

\begin{block}{По классификации}
Анализ псевдонимов в ОРС context-sensitive, flow-sensitive, field-sensitive (по умолчанию. Может быть и field-insensitive), flow-based.
\end{block}

\end{frame}

%-----------------------------------------------------------------------------------------
%-----------------------------------------------------------------------------------------
\section{Оптимизация существующего анализа}
%-----------------------------------------------------------------------------------------

\begin{frame}
\frametitle{Оптимизация существующего анализа}

\begin{block}{}
В существующем анализе было обнаружено два потенциальных места ускорения:
\end{block}

\begin{block}{В процедурной части анализа}
В случаях, когда в функции имеется множество расходящихся и сразу же сходящихся веток полный обход всех путей от входа до выхода занимает значительное время.
\end{block}

\begin{block}{В межпроцедурной части анализа}
Существующий алгоритм производит детальный анализ для каждого места вызова функции, что оказывается излишним в случае, если контекст вызовов не меняется.
\end{block}

\end{frame}

%-----------------------------------------------------------------------------------------

\begin{frame}
\frametitle{Оптимизация и результат}

\begin{block}{Оптимизация обхода ветвлений в управляющем графе}
Обход графа управления был модифицирован с применением некоторых идей из Steensgaard’s algorithm, чтобы уменьшить число путей, которые обходит анализ для построения результата.
\end{block}

\begin{block}{Полученное ускорение}
К сожалению, модифицированный обход не дал большого выигрыша на имеющихся бенчмарках, а на одном из них дал существенное замедление.
\end{block}

\end{frame}

%----------------------------------------------------------------------------------------

\begin{frame}
\frametitle{Оптимизация и результат}

\begin{block}{Оптимизация межпроцедурного анализа}
Полноценная реализация построения частично трансферных функций имела значительную трудоёмкость, поэтому было принято решение сначала реализовать простое кэширование результата для одних и тех же контекстов вызова функции.
\end{block}

\begin{block}{Полученное ускорение}
Для тестов небольших размеров скорость построения результата анализа почти не изменилась, но на достаточно сложных бенчмарках скорость удвоилась.
\end{block}

\end{frame}

%----------------------------------------------------------------------------------------

\begin{frame}
\frametitle{Оптимизация и результат}

\begin{block}{}

\begin{table}
\begin{tabular}{l | c | c | c | c }
Бенчмарк & Исходная реализация & Слияние & Кэширование & Оба \\
\hline \hline
linpack     & 193 & 189                    & 184                    & 183 \\
md          & 46  & 44                     & 40                     & 40\\ 
%whetstone   & 81  & 78                     & 75                     & 76\\
smith       & 211 & \textcolor{green}{117} & \textcolor{green}{98}  & 97\\
gazaryan    & 249 & 260                    & \textcolor{green}{92}  & 96\\
revrina     & 191 & 192                    & \textcolor{green}{67}  & 68\\
lbm         & 446 & \textcolor{red}{4474}  & 341                    & 4490\\
hirschberg  & 604 & 570                    & \textcolor{green}{309} & 306\\
hirschberg2 & 104 & 103                    & \textcolor{green}{50}  & 51\\
quagga      & 23  & \textcolor{green}{6}   & 22                     & 6
\end{tabular}
\end{table}
Среднее время работы на 20 запусках в миллисекундах.
\end{block}

\begin{block}{Тестовое окружение}
g++ 4.9.2, Ubuntu 15.04.\\
Intel(R) Core(TM) i7-4700MQ CPU @ 2.4 GHz (Up to 3.4 GHz)
\end{block}

\end{frame}

%----------------------------------------------------------------------------------------

\end{document}